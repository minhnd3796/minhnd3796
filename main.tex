\documentclass{article}
\usepackage[utf8]{inputenc}
\usepackage{hyperref}

\title{Master Thesis (draft)}
\author{Duc Minh Nguyen}
\date{October 2022}

\begin{document}

\maketitle

\tableofcontents

\section{Introduction}
\subsection{Context and Motivation}
\begin{itemize}
    \item Context: Federated Learning allows collective training of ML models
    without having the locally stored data to be transmitted to a central
    server. Kernel of Central server can be monitored, observed and customised
    without having to change the kernel with eBPF.
    \item Motivation: Apply eBPF in network communication of FL.
\end{itemize}

\subsection{Problem Statement}
Research results in FL have come only with multi-thread simulation and simple
model architecture and data, even with no simulated network communication.
Network communication of FL has large room to be explored, especially how to
handle high burst traffic on server side. Implement a defence mechanism of FL
as an in-network processing component of FL.

\subsection{Goals and Contributions}
\begin{itemize}
    \item Goals: Setup a simulation of FL with network communication and
    implement in-network processing component of FL.
    \item Contributions: Show that such simulation is possible with minimal
    overhead and sufficient to continue to explore the many directions of
    validations and computations of FL, especially experiments for defence
    mechanism of FL without having to build new FL framework or modify
    well-engineered and sophisticated frameworks.
\end{itemize}

\subsection{Thesis Outline}

\section{Background and Related Work}
\subsection{Machine Learning and Federated Learning}
\subsubsection{Machine Learning and Deep Learning}
Here I write very shortly about what is ML and DL.
\subsubsection{Distributed Machine Learning and Federated Learning}
The same for distributed ML and FL.
\subsubsection{Attacks against FL}
List and introduce some poisoning attacks (targeted, untargeted), inflation
attacks and sybil attacks.
\subsubsection{Defence Mechanisms for FL}
Short insight into some server-side and client-side defences, then lead the
readers to FoolsGold as an application for the captured submitted models.
\begin{itemize}
    \item Trusted Executed Environment
    Here I mention and introduce some of the client-side approaches similar to the
    one from Benedikt V\"olker
    \item FoolsGold
    Describe the essentials of the algorithms and its reported performance,
    point out my improvments and contributions based on the original paper.
\end{itemize}

\subsection{Distributed ML Framework}
\subsubsection{Overview}
General discussion on the landscape of FL framework, worth to mentioned:
Tensorflow Federated by Google and then guide to NVFlare and why I used it for
this network simulation.
\subsubsection{Remote Procedure Call with gRPC and ProtoBuf}
Introduction to RPC (distributed programming paradigm) in general and guide
the readers to gRPC of Google and its data format ProtoBuf whose data is
deserialised and transmitted on top of the DATA stream of HTTP/2 protocol.
\subsubsection{NVFlare - The FL Framework with Network Communication}
General introduction to system architecture of the NVIDIA Framework

\subsection{Docker}
\subsubsection{Docker container and its network stack}
General to Docker container and present the necessity of Docker in terms of its
private network and its Linux network stack when setting the network simulation.
\subsubsection{FL Simulation with Docker}
Why have to build Docker image myself for each type of machine (OS and CPU
architecture) to ensure the kernel-version capability of self-compiled,
otherwise, simulation for network communication of FL is not successful
because all of them are python version and kernel version dependent.

\subsection{eBPF}
\subsubsection{eBPF in kernel-customisation}
Arguments of utilisation of bcc with the raw socket in the experiment, although
there are myriads of options for using eBPF (XDP, GoLang, etc.)
\subsubsection{eBPF with Docker containers}
Describe how the Linux kernel of Linux Docker containers can benefit from eBPF.

\section{Design and Implementation}
\subsection{Design}
\subsubsection{Docker Image for the Simulation}
Short description about how to build the correct image for simulation for any
type of machines.
\subsubsection{In-network Submitted Model Capturing}
How to manually dissect the captured tcp packets to write kernel-side as well
as user-space eBPF script to capture the model on-the-wire with the ebpf raw
non-blocking socket.
\subsubsection{In-network Server-side Training Validation}
Describe how I incorporate FoolsGold to my simulation.

\subsection{Implementation}
\subsubsection{Kernel-space Script}
How the kernel-space script implements the eBPF packet filter to capture the
necessary packets on-the-wire in kernel space.
\subsubsection{User-space Script}
How the user-space Python script extract and deserialised the submitted model
the captured HTTP2 payload.
\subsubsection{Simplified FoolsGold}
Why "Simplified" FoolsGold and how it is suitably incorporated to the simulation

\section{Evaluation}
\subsection{eBPF Overhead}
Interpret the overhead of eBPF in various scenarios in which all submitted
parameters are dumped to disk.

\subsection{FoolsGold Performance}
Interpret the validation capabilities of FoolsGold in detecting no
malicious/sybil clients with different non-iid partitions of the FL simulation.

\subsection{Discussion of Experimental Results}

\section{Conclusion and Future Work}
\subsection{Conclusion}
\subsection{Future Work}

\end{document}
